%%
%% This is file `socius_conference_paper_merged.tex',
%%
\documentclass[sigconf]{acmart}

%%
%% \BibTeX command to typeset BibTeX logo in the docs
\AtBeginDocument{%
  \providecommand\BibTeX{{%
    \normalfont B\kern-0.5em{\scshape i\kern-0.25em b}\kern-0.8em\TeX}}}

%% Rights management information.
% \setcopyright{acmcopyright}
% \copyrightyear{2025}
% \acmYear{2025}
% \acmDOI{XXXXXXX.XXXXXXX}

%% These commands are for a PROCEEDINGS abstract or paper.
% \acmConference[Conference Name '25]{Make sure to enter the correct conference title from your rights confirmation email}{August 06--08, 2025}{Manila, Philippines}
% \acmBooktitle{Make sure to enter the correct conference title from your rights confirmation email, August 06--08, 2025, Manila, Philippines}
% \acmPrice{15.00}
% \acmISBN{978-1-4503-XXXX-X/18/06}
\acmConference[Socius '25]{Socius: A Dual-Agent System for Pragmatic Competence Training}{}{Manila, Philippines}
\acmBooktitle{Socius: A Dual-Agent System for Pragmatic Competence Training, Manila, Philippines}


%%
%% Submission ID.
%%\acmSubmissionID{123-A56-BU3}

% \renewcommand{\shortauthors}{Dimaculangan, et al.}
\setcopyright{none}
\settopmatter{printacmref=false}
\pagestyle{plain}
\renewcommand\footnotetextcopyrightpermission[1]{}


%%
%% The document itself.
\begin{document}

\title{Socius: A Dual-Agent Conversational System for Pragmatic Competence Training Grounded in Developmental and Cultural Frameworks}

\author{Aldwin Renzel P. Dimaculangan}
\email{aldwin_dimaculangan@dlsu.edu.ph}
\affiliation{%
  \institution{De La Salle University}
  \streetaddress{2401 Taft Avenue}
  \city{Manila}
  \country{Philippines}
}

\author{Miguel Alfonso D. Guerrero}
\email{miguel_guerrero@dlsu.edu.ph}
\affiliation{%
  \institution{De La Salle University}
  \streetaddress{2401 Taft Avenue}
  \city{Manila}
  \country{Philippines}
}

\author{Devon Jarek Y. Javier}
\email{devon_javier@dlsu.edu.ph}
\affiliation{%
  \institution{De La Salle University}
  \streetaddress{2401 Taft Avenue}
  \city{Manila}
  \country{Philippines}
}

\author{Luke M. Regalado}
\email{luke_regalado@dlsu.edu.ph}
\affiliation{%
  \institution{De La Salle University}
  \streetaddress{2401 Taft Avenue}
  \city{Manila}
  \country{Philippines}
}


\begin{abstract}
Young adults often lack safe, reflective environments to practice nuanced social communication, or pragmatic competence. Existing educational chatbots frequently lack the conversational naturalism or deep, culturally-aware evaluation frameworks necessary for effective training. This paper presents Socius, a chatbot for pragmatic competence training built on the AI Partner-AI Mentor (APAM) framework. Socius leverages a full Google Gemini architecture, utilizing a fast model for immersive roleplaying and a powerful model for deep analysis. A notable feature is our system's multi-pillar theoretical foundation, which grounds practice scenarios in a synthesis of Grice's Maxims, Havighurst's developmental tasks, Arnett's theory of emerging adulthood, and Filipino cultural values. We introduce a detailed, multi-level scoring framework that provides quantitative metrics on user performance across three core pragmatic domains. Preliminary findings detail the successful implementation of this architecture and report on a crucial experiment where fine-tuning a smaller LLM on the specialized GRICE dataset yielded pedagogically poor feedback. This result validated our successful pivot to a prompt-engineering approach with a more capable base model. This paper contributes the architecture and methodology of a theoretically robust, culturally-contextualized system for social skill training and offers preliminary insights into the trade-offs between fine-tuning and prompt-engineering for nuanced feedback generation.
\end{abstract}

\begin{CCSXML}
<ccs2012>
   <concept>
       <concept_id>10003120.10003121.10003124.10010866</concept_id>
       <concept_desc>Human-centered computing~HCI theory, concepts and models</concept_desc>
       <concept_significance>500</concept_significance>
       </concept>
   <concept>
       <concept_id>10010405.10010489.10010490</concept_id>
       <concept_desc>Applied computing~Computer-assisted instruction</concept_desc>
       <concept_significance>500</concept_significance>
       </concept>
 </ccs2012>
\end{CCSXML}

\ccsdesc[500]{Human-centered computing~HCI theory, concepts and models}
\ccsdesc[500]{Applied computing~Computer-assisted instruction}

\keywords{pragmatic competence, conversational AI, educational technology, dual-agent framework, Gemini API, Filipino cultural context}

\maketitle

\section{Introduction}
TPragmatic competence involves figuring out a statement’s implicit meaning, rather than explicit \cite{Zheng2021GRICE}. This skill is important in social communication, but it is a particularly challenging skill for young adults (aged 15-24) to master especially in social, academic, and professional settings. In the Philippine context, pragmatic competence is even more needed due to context being a vital part of communication culture, where values such as \textit{Pakikisama}, \textit{Hiya}, \textit{Utang na Loob}, \textit{Kapwa}, and \textit{Paggalang} shape conversational norms and is a contributing factor in how people structure their speech \cite{Garcia2013}. Despite having a need to improve on pragmatic competence, there is a lack of an accessible, judgment-free space to practice and refine this social skill.

To address this gap, a chatbot was developed that makes use of the Gemini large language model, paired with the AI Partner-AI Mentor (APAM) framework \cite{Yang2024}, which proposes a dual-agent structure: one that practices alongside the user (AI Partner) and one that provides personalized feedback based on performance with the Partner (AI Mentor). This paper details the design and preliminary findings of the developed chatbot.

This paper offers the following contributions:
\begin{enumerate}
    \item The design and implementation of a \textbf{dual-agent architecture} using Google Gemini for a natural roleplay experience, deep pragmatic analysis, and personalized feedback provision.
    \item A \textbf{theoretical framework for scenario generation} that synthesizes linguistic theory (Grice), developmental psychology (Havighurst, Arnett), and Filipino cultural values (Garcia).
    \item The development of a \textbf{multi-level, quantitative scoring rubric} for pragmatic competence that provides detailed, actionable metrics.
    \item \textbf{Preliminary findings} on the limitations of fine-tuning smaller models on narrow datasets for generating nuanced pedagogical feedback, validating architectural pivot.
    \item The architecture and methodology of a theoretically robust, culturally-contextualized system for social skill training.
\end{enumerate}

\section{Related Work}
Our work is situated at the intersection of conversational AI, educational technology, and sociolinguistics. This section reviews the landscape of relevant technologies and theories that inform the design of Socius.

\subsection{Conversational Agents for Social and Emotional Support}
The use of technology for social skills training has expanded from traditional computer-based programs to more immersive experiences like virtual reality and metaverse platforms \cite{Lee2023, Kolk2022}. Within this domain, chatbots have emerged as a particularly accessible and scalable solution. These systems typically fall into two categories: therapeutic support and communication skill training.

Therapeutic chatbots like Woebot leverage Cognitive Behavioral Therapy (CBT) to provide mental health support \cite{Fitzpatrick2017}, while systems like Amy use social stories to assist children with Autism Spectrum Disorder (ASD) \cite{Gagan2023}. In contrast, skill-training chatbots focus on improving conversational ability. For instance, Kulturo uses gamified, story-driven roleplay to teach Tagalog within a cultural context \cite{Aspecto2025}. 

A key differentiator in chatbot design is the interaction modality. The predecessor to this project, `SociusBot` \cite{Chen2022}, demonstrated the potential of a rule-based system for teens but was limited to multiple-choice inputs, which users found "robotic." This highlights a critical gap: the need for a system that combines the pedagogical structure of earlier designs with the conversational naturalism of modern LLMs, which systems like Kulturo have begun to explore. Socius is designed to fill this gap by offering free-text, dynamic interaction grounded in a robust theoretical framework.

\subsection{Comparison of Large Language Models for Conversational AI}
The choice of the underlying LLM is a critical architectural decision that dictates a chatbot's capabilities. OpenAI's GPT series, particularly GPT-4, has set a high benchmark for pragmatic reasoning, in some cases outperforming humans on tasks involving implicature detection \cite{Bojic2023}. These models have proven effective as flexible conversational agents adaptable to various roles \cite{Lee2024}.

However, the LLM landscape is diverse, and other models present compelling alternatives. Google's Gemini family of models has demonstrated strong performance, particularly in tasks requiring nuanced contextual understanding and sustained conversational coherence. A notable example is the ELEVATE virtual companion, a Gemini-powered mental health support tool for students, which showed significant iterative improvement in contextual understanding and empathetic engagement over five development cycles \cite{Angeline2024}. This demonstrated effectiveness in a supportive, educational context, combined with the architectural flexibility of using both fast (Gemini Flash) and powerful (Gemini Pro) models, made the Gemini API a viable and strategic choice as the foundational LLM for the Socius system's dual-agent architecture.

\subsection{Theoretical and Ethical Foundations}
Effective tool design must be grounded in both theory and ethical practice. Foundational studies in social communication highlight the period of young adulthood as a critical developmental stage for honing empathetic and functional communication skills, justifying our target demographic \cite{Guclu2016}. Research into social awkwardness provides a framework for recognizing and responding to conversational miscues, which is essential for the AI Mentor's feedback mechanism \cite{Clegg2012}.

Simultaneously, deploying AI in this domain requires careful ethical consideration. LLMs can inherit and perpetuate social prejudices and bias in their responses, due to their potential presence in the training data. These biased answers might cause discrimination and false information to be projected towards the user \cite{Zhou2024}. Additionally, the sensitive nature of user conversations and profile details necessitates robust privacy and data security measures. With this, it is important that the chatbot does not reinforce any biases nor compromise the privacy of its users. With these ethical challenges present in our methodology, the inclusion of a formal ethics review, informed consent, and data anonymization are designed to mitigate these risks and ensure the responsible development and deployment of Socius.

\begin{figure}
\centering
\includegraphics[width=\linewidth]{figures/[Socius] Architecture.png}
\caption{Figure 1. System Architecture of Socius}
\Description{A diagram showing the architecture of the Socius system. On the left is the User Interface, which connects to the Flask Backend in the center. The backend connects to a Supabase Database below it and to two AI Agents on the right: the AI Partner (Roleplay Engine) and the AI Mentor (Analyzer and Chat). Arrows indicate data flow between these components.}
\label{fig:architecture}
\end{figure}

\section{System Architecture}
Socius functions as an orchestrated system of distinct AI agents managed by a central backend. The architecture is designed to optimize both the speed of the user's practice session and the depth of the subsequent analysis.

\subsection{Backend: The Nervous System}
The backend serves as the central nervous system of the application, orchestrating the user's entire learning loop. It runs on a \textbf{Flask server} exposing RESTful API endpoints that coordinate data flow between the frontend, the database, and the AI agents.
\begin{itemize}
    \item \textbf{Session Management:} The server manages user sessions, ensuring isolation between different users and conversational contexts.
    \item \textbf{Data Persistence:} Through the \texttt{profile} module, the backend handles all database operations with Supabase. It manages CRUD operations for user profiles and conversation logs, ensuring that every interaction contributes to the user's longitudinal growth record.
    \item \textbf{API Abstraction:} The server abstracts the complexity of the AI integration through four primary endpoints:
    \begin{enumerate}
        \item \texttt{/start-session}: Initializes the chatbot instance and loads the selected scenario context.
        \item \texttt{/send-message}: Routes user input to the AI Partner for roleplay responses.
        \item \texttt{/end-session}: Triggers the analysis phase, generating feedback and saving data to the database.
        \item \texttt{/mentor-chat}: Enables the interactive Q\&A session with the AI Mentor.
    \end{enumerate}
\end{itemize}

\subsection{Core Logic: The Brain}
The core logic handles the three modes of the chatbot: "AI Partner," "AI Mentor Analyzer," and "AI Mentor Chat."

\subsubsection{Agent 1: AI Partner (Roleplay Engine)}
This agent functions as the real-time conversational partner. It is powered by \textbf{\texttt{gemini-2.5-flash}} to prioritize cost-effectiveness and high-frequency interaction speed.
\begin{itemize}
    \item \textbf{Workflow:} It receives the user message and conversation history. It constructs a prompt with the scenario context, character instructions, and previous dialogue.
    \item \textbf{Output:} It generates and streams a response to the UI, updating the history.
    \item \textbf{User Journey Connection:} This agent powers the live roleplay phase. The use of the "Flash" model ensures the conversation feels fluid and responsive, mimicking the speed of natural human dialogue to maintain immersion.
\end{itemize}

\subsubsection{Agent 2: AI Mentor Analyzer (Feedback Generation)}
This agent is responsible for post-conversation analysis. It is triggered only when the user ends a session and utilizes \textbf{\texttt{gemini-2.5-pro}} for its sophisticated reasoning capabilities.
\begin{itemize}
    \item \textbf{Workflow:} It receives the complete conversation transcript. It analyzes multiple dimensions including communication patterns, empathy levels, and scenario-specific goal achievement.
    \item \textbf{Output:} It generates structured feedback containing quantitative scores and qualitative explanations. Crucially, it primes the subsequent Chat Agent with this generated feedback.
    \item \textbf{User Journey Connection:} This represents the transition from practice to reflection. The "Pro" model is used here because evaluation is computationally more complex than generation; it must apply abstract theoretical concepts to specific user utterances.
\end{itemize}

\subsubsection{Agent 3: AI Mentor Chat (Interactive Q\&A)}
This agent facilitates personalized discussion about the user's performance. Like the Analyzer, it uses \textbf{\texttt{gemini-2.5-pro}}.
\begin{itemize}
    \item \textbf{Workflow:} It maintains a separate conversation history from the roleplay. When the user asks a question, the agent responds using the pre-loaded feedback context generated by the Analyzer.
    \item \textbf{User Journey Connection:} This facilitates the "debriefing" phase. By pre-priming the agent with the analysis, we eliminate the need to resend the full transcript for every follow-up question, significantly reducing token usage while allowing the user to ask clarifying questions about their results.
\end{itemize}

\section{Methodology and Theoretical Framework}
Our system is not just a technical implementation but a pedagogical tool grounded in a multi-pillar theoretical framework.

\subsection{Linguistic Contextualization through Negative Constraints}

\begin{itemize}
    \item \textbf{Negative Constraints}
    To ensure the system is culturally resonant for Filipino young adults, we moved beyond standard prompt engineering to implement rigorous "negative constraints" within the system instructions. Analysis of early prototypes revealed that generic LLMs often "break character" by explaining their code-switching. To counter this, the production system explicitly forbids the AI from providing parenthetical English translations (e.g., "Wala lang (It's nothing)") \cite{Zheng2021GRICE}.

    \item \textbf{Conversational Markers}
    Furthermore, the prompt enforces the appropriate use of specific Filipino markers. Rather than generic English fillers, the model is instructed to utilize particles such as \textit{naman}, \textit{nga}, \textit{diba}, and \textit{eh} to modulate tone and emphasis.

    \item \textbf{Targeted Code-Switching}
    Finally, the system is instructed to code-switch specifically for words carrying high emotional weight (e.g., \textit{nakakainis}, \textit{sayang}), mirroring the linguistic phenomenon where bilingual speakers revert to their L1 for affective expression.
\end{itemize}

\subsection{Theoretically-Grounded Scenario Generation}
Each scenario is synthesized from four core pillars to ensure they are realistic and pedagogically sound:
\begin{enumerate}
    \item \textbf{Arnett's Theory of Emerging Adulthood:} Provides the psychological context of instability and identity exploration \cite{Arnett2000}.
    \item \textbf{Havighurst's Developmental Tasks:} Provides the concrete life tasks (e.g., emotional independence, career preparation) \cite{Havighurst1972}.
    \item \textbf{Grice's Cooperative Principle:} Provides the linguistic challenge via the four maxims (Quantity, Quality, Relation, Manner) \cite{Grice1975}.
    \item \textbf{Filipino Cultural Values:} Provides the socio-cultural constraints (Pakikisama, Hiya, Utang na Loob) \cite{Garcia2013}.
\end{enumerate}

\subsubsection{Synthesizing Developmental Themes}
The synthesis of Arnett and Havighurst produces four distinct, recurring themes that structure our scenario database. Each theme has a defined conflict, formula, and goal.

\paragraph{Theme 1: Identity vs. Expectation Conflict}
This theme is built on the collision between a young adult's emerging sense of self (Arnett) and the life path prescribed for them by family or society (Havighurst).
\begin{itemize}
    \item \textbf{Conflict:} The user's personal desire, belief, or identity clashes with an external expectation from an authority figure. This represents the tension between \textit{Authenticity} and \textit{Obligation} (Utang na Loob/Paggalang).
    \item \textbf{Formula:} 1) The user is developing a personal goal, value, or identity. 2) An authority figure expresses a conflicting expectation or disappointment. 3) The situation forces the user to defend or conceal their choice.
    \item \textbf{Goal:} The user must practice validating the authority's perspective and expressing gratitude, while simultaneously and respectfully asserting their own identity and decisions.
\end{itemize}

\paragraph{Theme 2: Autonomy vs. Dependence Conflict}
This theme explores the frustration of being "in-between" (Arnett), where the user seeks adult responsibilities (Havighurst) but is still treated like a child.
\begin{itemize}
    \item \textbf{Conflict:} The user's attempt to exercise adult autonomy is challenged by a figure who still views them as dependent. This represents the tension between \textit{Independence} and \textit{Hierarchy}.
    \item \textbf{Formula:} 1) The user makes an adult decision (e.g., financial, travel). 2) A guardian discovers this and questions or overrules the decision. 3) The situation requires the user to justify their competence without being childishly defiant.
    \item \textbf{Goal:} The user must practice shifting the conversation from "obedience" to "responsibility," explaining their reasoning to demonstrate competence and reevaluate the relationship as adult-to-adult.
\end{itemize}

\paragraph{Theme 3: Ambiguity vs. Definition Conflict}
This theme explores the instability of relationships (Arnett) while trying to achieve stable partnerships (Havighurst).
\begin{itemize}
    \item \textbf{Conflict:} A relationship exists in an undefined "gray area," and a situation arises that forces the need for clarity. This represents the tension between \textit{Clarity} (Grice) and \textit{Fear of Loss} (Risking Pakikisama).
    \item \textbf{Formula:} 1) The user is in an ambiguous relationship (e.g., ``talking stage''). 2) A trigger event occurs (e.g., jealousy, holiday). 3) The situation forces a ``Define the Relationship'' (DTR) conversation.
    \item \textbf{Goal:} The user must practice expressing their own feelings and needs with honesty (Quality) and kindness (Manner), while creating a safe space for the other person to be honest.
\end{itemize}

\paragraph{Theme 4: Group vs. Self Conflict}
This theme centers on forming a personal value system (Havighurst) when it conflicts with the established norms of a peer group (Arnett's Identity Exploration).
\begin{itemize}
    \item \textbf{Conflict:} The user's personal values are violated by the actions or words of their friends. This represents the tension between \textit{Group Harmony} (Pakikisama) and \textit{Personal Integrity} (Kapwa/Hiya).
    \item \textbf{Formula:} 1) The peer group does something unethical or wrong. 2) Silence implies agreement. 3) The situation forces the user to choose between staying silent or intervening.
    \item \textbf{Goal:} The user must practice intervening in a way that minimizes defensiveness. They learn to redirect the conversation or state their own feelings to de-escalate the conflict.
\end{itemize}

\subsection{Pragmatic Competence Scoring Framework}
To transform subjective sociolinguistic analysis into measurable data, the evaluation engine utilizes the Controlled Generation capabilities of the Gemini API. Rather than relying on probabilistic text generation, we enforce a strict JSON Output Schema within the model configuration. This constrains the AI Mentor to populate a pre-defined data structure, ensuring that every session yields machine-parsable metrics for longitudinal tracking. The schema requires the model to output both a numerical score (1-5) and a specific textual justification for criteria across three core domains:

\subsubsection{Category 1: Intention Coherence}
This category measures how well the user’s response fits into the logical flow of the conversation.
\begin{itemize}
    \item \textbf{Criterion 1: Speech Act Appropriateness:} Did the user use the correct type of utterance (e.g., request, refusal, apology) for the situation?
    \item \textbf{Criterion 2: Discourse Relation Clarity:} Is the connection between the user’s response and the previous turn logical and clear?
\end{itemize}

\subsubsection{Category 2: Social \& Affective Nuance}
This category measures the user’s ability to navigate the social and emotional context, particularly relevant to Filipino values.
\begin{itemize}
    \item \textbf{Criterion 3: Formality Level:} Did the user adopt the correct register (casual vs. formal) given the power dynamics?
    \item \textbf{Criterion 4: Informativeness:} Did the user provide the right amount of information (Grice’s Maxim of Quantity) without over-explaining or being curt?
    \item \textbf{Criterion 5: Emotional Awareness:} Did the tone of the response demonstrate empathy and awareness of the partner’s emotional state?
\end{itemize}

\subsubsection{Category 3: Communicative Effectiveness}
This category measures the functional success of the communication.
\begin{itemize}
    \item \textbf{Criterion 6: Implicature Clarity:} When being indirect, was the implied meaning effectively conveyed and understood?
    \item \textbf{Criterion 7: Goal Achievement:} Did the response successfully move the user closer to their conversational goal (e.g., successfully refusing a request without causing offense)?
\end{itemize}

\subsection{Holistic Mentor Functionality}
Beyond single-session feedback, Socius includes an advanced "Holistic Mentor" feature. This function retrieves a user's entire conversation history from the database. Rather than relying on a separate summarization algorithm, we leverage the large context window of \texttt{gemini-2.5-pro} to aggregate the raw logs and score data from all past sessions directly. This allows the Mentor to identify recurring patterns (e.g., "You consistently score low on Formality with authority figures") and track progress over time, offering high-level strategic advice on the user's overall communication trajectory.

\section{Preliminary Findings}
\subsection{Implementation Status}
The core Socius architecture—including the dual-agent Gemini chatbots, Flask backend, and Supabase logging—has been successfully implemented.

\subsection{Case Study: The Limits of Fine-Tuning on Small, Specialized Datasets}
Our initial methodology proposed fine-tuning a smaller, open-source model (Mistral-7B) on the GRICE dataset \cite{Zheng2021GRICE} to create a specialized AI Mentor.

\textbf{Hypothesis:} We hypothesized that fine-tuning would create an expert model adept at identifying Gricean maxim violations and providing targeted feedback.

\textbf{Results:} The fine-tuned model performed poorly in a pedagogical context. Its feedback was overly literal, focusing only on direct maxim violations from the dataset. It lacked a supportive, encouraging tone and struggled to handle the natural messiness and complexity of real conversation that fell outside the dataset's narrow scope. The responses were often "direct and bad," failing to provide constructive guidance.

\textbf{Analysis:} We conclude that the GRICE dataset, while invaluable for its intended task of implicature recovery, is too small and specialized for training a general-purpose communication coach. The model overfitted to the dataset's structure, resulting in a brittle and unhelpful feedback generator.

\subsection{The Viability of a Prompt-Engineered Approach}
Following the fine-tuning experiment, we pivoted our strategy to prompt engineering with a powerful, generalist foundation model (Gemini 2.5 Pro). Initial tests of this new approach have been highly successful. By providing the model with the conversation transcript, the detailed scoring rubric, and a carefully constructed meta-prompt instructing it to act as an encouraging, insightful coach, the AI Mentor now generates feedback that is significantly superior. The feedback is nuanced, contextually relevant, supportive in tone, and accurately populates the structured JSON scoring object.

\section{Future Work and Remaining Tasks}
Our primary focus moving forward is the user validation phase of the project. We plan to recruit approximately 20 participants within our target demographic of 15-24 years old. Our methodology will involve a pre-assessment survey to establish a baseline, a three-week period of interaction with the Socius chatbot, and post-interaction surveys and interviews to gather both quantitative and qualitative data.

The data analysis will focus on two areas: quantitatively measuring any changes in pragmatic scores across multiple sessions, and qualitatively analyzing user feedback to assess the system's usability, engagement, and perceived effectiveness. Insights from this phase will be used to refine the AI Mentor's prompts and the scenarios themselves. This study has undergone a full ethics review to ensure participant data is handled responsibly and all interactions are anonymized for analysis.

\section{Conclusion}
This paper has presented the architecture and methodology of Socius, a dual-agent conversational system designed to train pragmatic competence. We have detailed a novel approach that integrates linguistic theory with developmental psychology and Filipino cultural values to create a robust and contextually relevant learning experience. Our key preliminary finding highlights a critical insight for the development of educational AI: for complex, pedagogical tasks like providing communication feedback, the advanced reasoning capabilities of a large, generalist model guided by sophisticated prompt engineering can significantly outperform smaller models fine-tuned on narrow, specialized datasets. Future work will focus on validating this promising system with real users to measure its effectiveness in enhancing real-world social communication skills.

\begin{acks}
The authors would like to thank their adviser, Dr. Ethel Ong, and the panel members, Mr. Edward Tighe and Ms. Jackylyn Beredo, for their invaluable guidance and feedback throughout this research.
\end{acks}

\bibliographystyle{ACM-Reference-Format}
\begin{thebibliography}{99}

\bibitem{Angeline2024}
R. Angeline, S. Roshini, M. A. Sajith, and M. Shasti. 2024. ELEVATE: An AI-Driven Virtual Companion for Nurturing Mental Health Support for Students. In \textit{2024 2nd International Conference on Advances in Computation, Communication and Information Technology (ICAICCIT)}. 1389-1394.

\bibitem{Arnett2000}
Jeffrey Jensen Arnett. 2000. Emerging adulthood: A theory of development from the late teens through the twenties. \textit{American psychologist}, 55, 5 (2000), 469.

\bibitem{Aspecto2025}
D. Aspecto, B. Lu, P. Pacheco, and J. Uytanlet. 2025. KulTuro: Exploring the use of a role-play based storytelling chatbot for Tagalog language education. Undergraduate Thesis, De La Salle University.

\bibitem{Bojic2023}
L. Bojic, P. Kovacevic, and M. Cabarkapa. 2023. GPT-4 Surpassing Human Performance in Linguistic Pragmatics. \textit{arXiv preprint arXiv:2312.09545}.

\bibitem{Chen2022}
A. C. Chen, D. J. Y. Javier, E. A. Panugayan, L. M. Regalado, and R. R. Resureccion. 2022. Project sociusbot: Development of a human-like interactive chatbot for social communication. Project paper, De La Salle University Integrated School.

\bibitem{Clegg2012}
J. W. Clegg. 2012. The experience of social awkwardness. \textit{Qualitative Psychology}, 20 (2012), 22-45.

\bibitem{Fitzpatrick2017}
K. K. Fitzpatrick, A. Darcy, and M. Vierhile. 2017. Delivering cognitive behavior therapy to young adults with symptoms of depression and anxiety using a fully automated conversational agent (woebot): A randomized controlled trial. \textit{JMIR Mental Health}, 4, 2 (2017), e19.

\bibitem{Gagan2023}
I. T. Gagan, M. A. M. Matias, I. Tan, C. M. Vinco, and E. Ong. 2023. Preparing Children with Level 1 ASD for Social Interactions through Storytelling with Amy: An Exploratory Study. In \textit{Extended Abstracts of the 2023 CHI Conference on Human Factors in Computing Systems}. 1--7.

\bibitem{Garcia2013}
F. C. N. Garcia. 2013. A study on the communication styles of Filipino young adults. \textit{Journal of Arts and Humanities}, 2, 6 (2013), 25-33.

\bibitem{Grice1975}
H. P. Grice. 1975. Logic and conversation. In \textit{Syntax and semantics, 3: Speech acts}, P. Cole and J. Morgan (Eds.). Academic Press, 41--58.

\bibitem{Guclu2016}
M. Guclu. 2016. The effects of communication skills education on university students. \textit{Journal of Higher Education and Science}, 6, 2 (2016), 181-187.

\bibitem{Havighurst1972}
Robert J. Havighurst. 1972. \textit{Developmental tasks and education}. David McKay Co.

\bibitem{Kolk2022}
A. van der Kolk, B. J. H. van der Vlist, and E. A. M. van der Veen. 2022. Social skills training for children with neurological disorders using multitouch-multiuser tabletops and virtual reality: A pilot study. \textit{JMIR serious games}, 10, 2 (2022), e34190.

\bibitem{Lee2024}
J. H. Lee, D. Shin, and Y. Hwang. 2024. Investigating the capabilities of large language model-based task-oriented dialogue chatbots from a learner's perspective. \textit{System}, 127, (2024), 103538.

\bibitem{Lee2023}
J. H. Lee, T. S. Lee, S. Y. Yoo, et al. 2023. Metaverse-based social skills training programme for children with autism spectrum disorder to improve social interaction ability: an open-label, single-centre, randomised controlled pilot trial. \textit{EClinicalMedicine}, 61, (2023), 102072.

\bibitem{Yang2024}
D. Yang, C. Ziems, W. Held, O. Shaikh, M. S. Bernstein, and J. Mitchell. 2024. Social Skill Training with Large Language Models. \textit{arXiv preprint arXiv:2404.04204}.

\bibitem{Zheng2021GRICE}
Z. Zheng, S. Qiu, L. Fan, Y. Zhu, and S.-C. Zhu. 2021. GRICE: A Grammar-based Dataset for Recovering Implicature and Conversational Reasoning. In \textit{Findings of the Association for Computational Linguistics: ACL-IJCNLP 2021}. 2074--2085.

\bibitem{Zhou2024}
J. Zhou, H. Müller, A. Holzinger, and F. Chen. 2024. Ethical chatgpt: Concerns, challenges, and commandments. \textit{Electronics}, 13, 17 (2024), 3417.

\end{thebibliography}

\end{document}
\endinput