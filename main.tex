%%
%% This is file `socius_conference_paper_v3.tex',
%%
\documentclass[sigconf]{acmart}

%%
%% \BibTeX command to typeset BibTeX logo in the docs
\AtBeginDocument{%
  \providecommand\BibTeX{{%
    \normalfont B\kern-0.5em{\scshape i\kern-0.25em b}\kern-0.8em\TeX}}}

%% Rights management information.
\setcopyright{acmcopyright}
\copyrightyear{2025}
\acmYear{2025}
\acmDOI{XXXXXXX.XXXXXXX}

%% These commands are for a PROCEEDINGS abstract or paper.
\acmConference[Conference Name '25]{Make sure to enter the correct conference title from your rights confirmation email}{August 06--08, 2025}{Manila, Philippines}
\acmBooktitle{Make sure to enter the correct conference title from your rights confirmation email, August 06--08, 2025, Manila, Philippines}
\acmPrice{15.00}
\acmISBN{978-1-4503-XXXX-X/18/06}


%%
%% Submission ID.
%%\acmSubmissionID{123-A56-BU3}

\renewcommand{\shortauthors}{Dimaculangan, et al.}

%%
%% The document itself.
\begin{document}

\title{Socius: A Dual-Agent Conversational System for Pragmatic Competence Training Grounded in Developmental and Cultural Frameworks}

\author{Aldwin Renzel P. Dimaculangan}
\email{aldwin_dimaculangan@dlsu.edu.ph}
\affiliation{%
  \institution{De La Salle University}
  \streetaddress{2401 Taft Avenue}
  \city{Manila}
  \country{Philippines}
}

\author{Miguel Alfonso D. Guerrero}
\email{miguel_guerrero@dlsu.edu.ph}
\affiliation{%
  \institution{De La Salle University}
  \streetaddress{2401 Taft Avenue}
  \city{Manila}
  \country{Philippines}
}

\author{Devon Jarek Y. Javier}
\email{devon_javier@dlsu.edu.ph}
\affiliation{%
  \institution{De La Salle University}
  \streetaddress{2401 Taft Avenue}
  \city{Manila}
  \country{Philippines}
}

\author{Luke M. Regalado}
\email{luke_regalado@dlsu.edu.ph}
\affiliation{%
  \institution{De La Salle University}
  \streetaddress{2401 Taft Avenue}
  \city{Manila}
  \country{Philippines}
}


\begin{abstract}
Young adults often lack safe, reflective environments to practice nuanced social communication, or pragmatic competence. Existing educational chatbots frequently lack the conversational naturalism or deep, culturally-aware evaluation frameworks necessary for effective training. This paper presents Socius, a chatbot for pragmatic competence training built on the AI Partner-AI Mentor (APAM) framework. Socius leverages a full Google Gemini architecture, utilizing a fast model for immersive roleplaying and a powerful model for deep analysis. A notable feature is our system's multi-pillar theoretical foundation, which grounds practice scenarios in a synthesis of Grice's Maxims, Havighurst's developmental tasks, Arnett's theory of emerging adulthood, and Filipino cultural values. We introduce a detailed, multi-level scoring framework that provides quantitative metrics on user performance across three core pragmatic domains. Preliminary findings detail the successful implementation of this architecture and report on a crucial experiment where fine-tuning a smaller LLM on the specialized GRICE dataset yielded pedagogically poor feedback. This result validated our successful pivot to a prompt-engineering approach with a more capable base model. This paper contributes the architecture and methodology of a theoretically robust, culturally-contextualized system for social skill training and offers preliminary insights into the trade-offs between fine-tuning and prompt-engineering for nuanced feedback generation.
\end{abstract}

\begin{CCSXML}
<ccs2012>
   <concept>
       <concept_id>10003120.10003121.10003124.10010866</concept_id>
       <concept_desc>Human-centered computing~HCI theory, concepts and models</concept_desc>
       <concept_significance>500</concept_significance>
       </concept>
   <concept>
       <concept_id>10010405.10010489.10010490</concept_id>
       <concept_desc>Applied computing~Computer-assisted instruction</concept_desc>
       <concept_significance>500</concept_significance>
       </concept>
 </ccs2012>
\end{CCSXML}

\ccsdesc[500]{Human-centered computing~HCI theory, concepts and models}
\ccsdesc[500]{Applied computing~Computer-assisted instruction}

\keywords{pragmatic competence, conversational AI, educational technology, dual-agent framework, Gemini API, Filipino cultural context}

\maketitle

\section{Introduction}
The ability to navigate complex social interactions, often termed pragmatic competence, involves understanding what is implied 
rather than what is explicitly stated \cite{Zheng2021GRICE}. This skill is particularly challenging for young adults (aged 15-24) who are 
concurrently navigating new academic, professional, and social environments. In the Philippine context, this challenge is amplified by a high-context
communication culture, where values such as \textit{pakikisama} (harmony), \textit{hiya} (shame/propriety), and \textit{utang na loob} 
(debt of gratitude) profoundly shape conversational norms \cite{Garcia2013}. Despite the clear need, there is a scarcity of accessible, 
judgment-free tools for practicing these skills.

To address this gap, we adopt the AI Partner-AI Mentor (APAM) framework \cite{Yang2024}, which proposes a dual-agent structure: one AI for safe, 
realistic practice (the Partner) and another for expert, reflective feedback (the Mentor). This paper details the design and preliminary findings of 
Socius, an evolution of a previous concept \cite{Chen2022} that has been re-architected to leverage the advanced capabilities of the Google 
Gemini family of large language models (LLMs).

This paper offers several key contributions:
\begin{enumerate}
    \item The design and implementation of a \textbf{dual-agent architecture} using Google Gemini for both naturalistic roleplay and deep pragmatic analysis.
    \item A \textbf{theoretical framework for scenario generation} that synthesizes linguistic theory (Grice), developmental psychology (Havighurst, Arnett), and Filipino cultural values (Garcia).
    \item The development of a \textbf{multi-level, quantitative scoring rubric} for pragmatic competence that provides detailed, actionable metrics.
    \item \textbf{Preliminary findings} on the limitations of fine-tuning smaller models on narrow datasets for generating nuanced pedagogical feedback, influencing our architectural pivot.
\end{enumerate}

\section{Related Work}
Our work is situated at the intersection of conversational AI, educational technology, and sociolinguistics. This section reviews the landscape of 
relevant technologies and theories that inform the design of Socius.

\subsection{Conversational Agents for Social and Emotional Support}
The use of technology for social skills training has expanded from traditional computer-based programs to more immersive experiences 
like virtual reality and metaverse platforms \cite{Lee2023, Kolk2022}. Within this domain, chatbots have emerged as a particularly accessible
 and scalable solution. These systems typically fall into two categories: therapeutic support and communication skill training.

Therapeutic chatbots like Woebot leverage Cognitive Behavioral Therapy (CBT) to provide mental health support \cite{Fitzpatrick2017}, while systems 
like Amy use social stories to assist children with Autism Spectrum Disorder (ASD) \cite{Gagan2023}. In contrast, skill-training chatbots focus on
 improving conversational ability. For instance, Kulturo uses gamified, story-driven roleplay to teach Tagalog within a cultural context \cite{Aspecto2025}. 

A key differentiator in chatbot design is the interaction modality. The predecessor to this project, `SociusBot` \cite{Chen2022},
 demonstrated the potential of a rule-based system for teens but was limited to multiple-choice inputs, which users found "robotic." 
 This highlights a critical gap: the need for a system that combines the pedagogical structure of earlier designs with the conversational 
 naturalism of modern LLMs, which systems like Kulturo have begun to explore. Socius is designed to fill this gap by offering free-text, 
 dynamic interaction grounded in a theoretical framework.

\subsection{Comparison of Large Language Models for Conversational AI}
The choice of the underlying LLM is a critical architectural decision that dictates a chatbot's capabilities. OpenAI's GPT series, particularly GPT-4, 
has set a high benchmark for pragmatic reasoning, in some cases outperforming humans on tasks involving implicature detection \cite{Bojic2023}. 
These models have proven effective as flexible conversational agents adaptable to various roles \cite{Lee2024}.

However, the LLM landscape is diverse, and other models present alternatives. Google's Gemini family of models has demonstrated strong
performance, particularly in tasks requiring nuanced contextual understanding and sustained conversational coherence. A notable example is the 
ELEVATE virtual companion, a Gemini-powered mental health support tool for students, which showed iterative differences in contextual 
understanding and empathetic engagement over five development cycles \cite{Angeline2024}. This demonstrated effectiveness in a supportive,
educational context, combined with the architectural flexibility of using both fast (Gemini Flash) and powerful (Gemini Pro) models, made
the Gemini API a viable and strategic choice as the foundational LLM for the Socius system's dual-agent architecture.

\subsection{Theoretical and Ethical Foundations}
Effective tool design must be grounded in both theory and ethical practice. Foundational studies in social communication highlight
the period of young adulthood as a critical developmental stage for honing empathetic and functional communication skills, 
justifying our target demographic \cite{Guclu2016}. Research into social awkwardness provides a framework for recognizing and 
responding to conversational miscues, which is essential for the AI Mentor's feedback mechanism \cite{Clegg2012}.

Simultaneously, deploying AI in this domain requires ethical consideration. LLMs can inherit and perpetuate biases from their training data, 
and the sensitive nature of user conversations necessitates privacy and data security measures \cite{Zhou2024}. Our methodology, 
which includes a formal ethics review and data anonymization, is designed to mitigate these risks and ensure the responsible 
development and deployment of Socius.

\section{The Socius System: Architecture and Methodology}
Socius is implemented as a full-stack application with a Python Flask backend, a Supabase database, and a web-based frontend. 
The core innovation lies in its dual-agent Gemini architecture and its theoretically-grounded methodology.

\subsection{System Architecture}

% FIGURE 1: System Architecture Diagram
Based on our review of current LLMs, the Gemini family was selected for its strong performance in contextual reasoning and 
its architectural flexibility. The system employs two distinct models to fulfill the roles of the APAM framework:
\begin{itemize}
    \item \textbf{AI Partner:} Handled by a fast model, \texttt{gemini-flash-latest}, to engage the user in responsive, real-time roleplaying.
    \item \textbf{AI Mentor:} Handled by a powerful model, \texttt{gemini-2.5-pro}, for the complex reasoning required for scoring and feedback generation.
\end{itemize}
The backend manages session state, user profiles, and conversation logs, which are crucial for tracking progress and 
enabling the holistic mentor functionality.

% BEGIN HEAVILY REVISED SECTION: Detailed Theoretical Framework
\subsection{Theoretically-Grounded Scenario Generation}
Our scenarios are not arbitrary conversational prompts but are systematically constructed from a multi-pillar theoretical foundation to 
ensure they are realistic, culturally resonant, and pedagogically sound. Each scenario is informed by four core pillars:
\begin{description}
    \item[Arnett's Theory of Emerging Adulthood] provides the \textit{psychological context} of instability, identity exploration, and ambiguity that defines the user's life stage \cite{Arnett2000}.
    \item[Havighurst's Developmental Tasks] provides the concrete \textit{life task} or social challenge the user must navigate, such as forming mature relationships or achieving emotional independence \cite{Havighurst1972}.
    \item[Grice's Maxims] provides the \textit{linguistic challenge}, focusing on the use and interpretation of conversational implicatures and the cooperative principle \cite{Grice1975}.
    \item[Filipino Cultural Values] provide the \textit{socio-cultural constraints}, embedding values like \textit{pakikisama} (harmony) and \textit{paggalang} (respect) into the scenario's social dynamics \cite{Garcia2013}.
\end{description}

\subsubsection{Synthesizing Developmental Themes}
The core narrative conflict in our scenarios is generated by the collision between the psychological state of an emerging adult (Arnett) 
and the concrete life tasks they are expected to master (Havighurst). This synthesis produces four distinct, recurring themes that are 
relevant to our target users.

\paragraph{Theme 1: Identity vs. Expectation Conflict}
This theme is built on the collision between a young adult's emerging sense of self (Arnett) and the life path prescribed 
for them by society or family (Havighurst). The user's personal desire clashes with an external expectation from an authority figure, 
forcing them to practice validating the other's perspective while respectfully asserting their own identity. This pits the general 
idea of \textit{Authenticity} (Arnett's Identity Exploration) against \textit{Obligation} (Filipino values of Utang na Loob and Paggalang).

\paragraph{Theme 2: Autonomy vs. Dependence Conflict}
This theme explores the frustration of being "in-between" (Arnett), where a user seeks adult responsibilities (Havighurst) 
but is still treated as dependent or incapable by a guardian figure. The user's attempt to exercise adult autonomy is challenged, 
requiring them to justify their competence without being childishly defiant, shifting the conversation from "obedience" to "responsibility." 
This conflict highlights \textit{Independence} (Arnett's Self-Focused Age) versus \textit{Hierarchy} (Paggalang, Family Roles).

\paragraph{Theme 3: Ambiguity vs. Definition Conflict}
This theme addresses the instability of relationships (Arnett) while trying to achieve stable partnerships (Havighurst). A 
friendship or romantic relationship exists in an undefined "gray area," and a situation arises that forces the need for clarity. The user must 
practice expressing their feelings and needs with honesty (Grice's Maxim of Quality) and kindness (Manner), creating a safe space for the 
other person. This scenario stages a conflict between \textit{Clarity} (Grice's Maxims) and the \textit{Fear of Loss} (risking Pakikisama).

\paragraph{Theme 4: Group vs. Self Conflict}
This theme centers on forming a personal value system (Havighurst) when it conflicts with the established norms of a 
peer group (Arnett's Identity Exploration). The user's personal values are violated by the actions of peers, forcing them to choose 
between staying silent (implying agreement) and intervening. The goal is to practice intervening in a way that minimizes defensiveness, 
de-escalates conflict, and maintains personal integrity. This theme explores the tension between \textit{Group Harmony} (Pakikisama) and 
\textit{Personal Integrity} (Kapwa, Hiya).
% END HEAVILY REVISED SECTION

\subsection{Pragmatic Competence Scoring Framework}
The cornerstone of our evaluation is a detailed, quantitative scoring framework. After each session, the AI Mentor is prompted to 
analyze the transcript and return a structured JSON object. This rubric assesses performance across three main categories:
\begin{itemize}
    \item \textbf{Intention Coherence:} Evaluates the appropriateness of speech acts and the clarity of discourse relations.
    \item \textbf{Social \& Affective Nuance:} Assesses the user's management of formality, informativeness, and emotional awareness in context.
    \item \textbf{Communicative Effectiveness:} Measures the clarity of the user's implicatures and their success in achieving the conversational goal of the scenario.
\end{itemize}
Each category includes several sub-scores and a textual justification, providing users with a multi-faceted view of their performance.

\subsection{Holistic Mentor Functionality}
Beyond single-session feedback, Socius includes an advanced "Holistic Mentor" feature. This function retrieves a user's entire 
performance history from the database, including scores and logs from all past scenarios. This longitudinal data is summarized and 
provided as context to the AI Mentor, allowing it to identify recurring patterns, track progress over time, and offer high-level 
strategic advice on the user's overall communication style.

\section{Preliminary Findings}
\subsection{Implementation Status}
The core Socius architecture has been successfully implemented. This includes the dual-agent Gemini-powered chatbot, the Flask backend API, 
user profile and conversation logging to the Supabase database, and the prompt-based generation of scores and feedback according to our JSON rubric.

\subsection{Case Study: The Limits of Fine-Tuning on Small, Specialized Datasets}
Our initial methodology proposed fine-tuning a smaller, open-source model (Mistral-7B) on the GRICE dataset \cite{Zheng2021GRICE} 
to create a specialized AI Mentor.

\textbf{Hypothesis:} We hypothesized that fine-tuning would create an expert model adept at identifying Gricean maxim violations and providing 
targeted feedback.

\textbf{Results:} The fine-tuned model performed poorly in a pedagogical context. Its feedback was overly literal, focusing only on direct 
maxim violations from the dataset. It lacked a supportive, encouraging tone and struggled to handle the natural messiness and complexity of 
eal conversation that fell outside the dataset's narrow scope. The responses were often "direct and bad," failing to provide constructive guidance.

\textbf{Analysis:} We conclude that the GRICE dataset, while invaluable for its intended task of implicature recovery, is too small and 
specialized for training a general-purpose communication coach. The model overfitted to the dataset's structure, resulting in a brittle and 
unhelpful feedback generator.

\subsection{The Viability of a Prompt-Engineered Approach}
Following the fine-tuning experiment, we pivoted our strategy to prompt engineering with a powerful, generalist foundation model 
(Gemini 2.5 Pro). Initial tests of this new approach have been highly successful. By providing the model with the conversation transcript, 
the detailed scoring rubric, and a carefully constructed meta-prompt instructing it to act as an encouraging, insightful coach, the AI Mentor 
now generates feedback that is aligned closesly to the framework. The feedback is nuanced, contextually relevant, supportive in tone, and accurately 
populates the structured JSON scoring object.

\section{Future Work and Remaining Tasks}
Our primary focus moving forward is the user validation phase of the project. We plan to recruit approximately 20 participants within our 
target demographic of 15-24 years old. Our methodology will involve a pre-assessment survey to establish a baseline, a three-week period of 
interaction with the Socius chatbot, and post-interaction surveys and interviews to gather both quantitative and qualitative data.

The data analysis will focus on two areas: quantitatively measuring any changes in pragmatic scores across multiple sessions, and qualitatively 
analyzing user feedback to assess the system's usability, engagement, and perceived effectiveness. Insights from this phase will be used to
refine the AI Mentor's prompts and the scenarios themselves. This study has undergone a full ethics review to ensure 
participant data is handled responsibly and all interactions are anonymized for analysis.

\section{Conclusion}
This paper has presented the architecture and methodology of Socius, a dual-agent conversational system designed to 
train pragmatic competence. We have detailed an approach that integrates linguistic theory with developmental psychology 
and Filipino cultural values to create a robust and contextually relevant learning experience. Our key preliminary finding highlights 
a critical insight for the development of educational AI: for complex, pedagogical tasks like providing communication feedback, the advanced 
reasoning capabilities of a large, generalist model guided by prompt engineering can provide more focused responses and analysis than smaller 
models fine-tuned on narrow, specialized datasets. Future work will focus on validating this system with real users to measure its 
effectiveness in enhancing real-world social communication skills.

\begin{acks}
The authors would like to thank their adviser, Dr. Ethel Ong, and the panel members, Mr. Edward Tighe and Ms. Jackylyn Beredo, 
for their invaluable guidance and feedback throughout this research.
\end{acks}

\bibliographystyle{ACM-Reference-Format}
\begin{thebibliography}{99}

\bibitem{Angeline2024}
R. Angeline, S. Roshini, M. A. Sajith, and M. Shasti. 2024. ELEVATE: An AI-Driven Virtual Companion for Nurturing Mental Health Support for Students. In \textit{2024 2nd International Conference on Advances in Computation, Communication and Information Technology (ICAICCIT)}. 1389-1394.

\bibitem{Arnett2000}
Jeffrey Jensen Arnett. 2000. Emerging adulthood: A theory of development from the late teens through the twenties. \textit{American psychologist}, 55, 5 (2000), 469.

\bibitem{Aspecto2025}
D. Aspecto, B. Lu, P. Pacheco, and J. Uytanlet. 2025. KulTuro: Exploring the use of a role-play based storytelling chatbot for Tagalog language education. Undergraduate Thesis, De La Salle University.

\bibitem{Bojic2023}
L. Bojic, P. Kovacevic, and M. Cabarkapa. 2023. GPT-4 Surpassing Human Performance in Linguistic Pragmatics. \textit{arXiv preprint arXiv:2312.09545}.

\bibitem{Chen2022}
A. C. Chen, D. J. Y. Javier, E. A. Panugayan, L. M. Regalado, and R. R. Resureccion. 2022. Project sociusbot: Development of a human-like interactive chatbot for social communication. Project paper, De La Salle University Integrated School.

\bibitem{Clegg2012}
J. W. Clegg. 2012. The experience of social awkwardness. \textit{Qualitative Psychology}, 20 (2012), 22-45.

\bibitem{Fitzpatrick2017}
K. K. Fitzpatrick, A. Darcy, and M. Vierhile. 2017. Delivering cognitive behavior therapy to young adults with symptoms of depression and anxiety using a fully automated conversational agent (woebot): A randomized controlled trial. \textit{JMIR Mental Health}, 4, 2 (2017), e19.

\bibitem{Gagan2023}
I. T. Gagan, M. A. M. Matias, I. Tan, C. M. Vinco, and E. Ong. 2023. Preparing Children with Level 1 ASD for Social Interactions through Storytelling with Amy: An Exploratory Study. In \textit{Extended Abstracts of the 2023 CHI Conference on Human Factors in Computing Systems}. 1--7.

\bibitem{Garcia2013}
F. C. N. Garcia. 2013. A study on the communication styles of Filipino young adults. \textit{Journal of Arts and Humanities}, 2, 6 (2013), 25-33.

\bibitem{Grice1975}
H. P. Grice. 1975. Logic and conversation. In \textit{Syntax and semantics, 3: Speech acts}, P. Cole and J. Morgan (Eds.). Academic Press, 41--58.

\bibitem{Guclu2016}
M. Guclu. 2016. The effects of communication skills education on university students. \textit{Journal of Higher Education and Science}, 6, 2 (2016), 181-187.

\bibitem{Havighurst1972}
Robert J. Havighurst. 1972. \textit{Developmental tasks and education}. David McKay Co.

\bibitem{Kolk2022}
A. van der Kolk, B. J. H. van der Vlist, and E. A. M. van der Veen. 2022. Social skills training for children with neurological disorders using multitouch-multiuser tabletops and virtual reality: A pilot study. \textit{JMIR serious games}, 10, 2 (2022), e34190.

\bibitem{Lee2024}
J. H. Lee, D. Shin, and Y. Hwang. 2024. Investigating the capabilities of large language model-based task-oriented dialogue chatbots from a learner's perspective. \textit{System}, 127, (2024), 103538.

\bibitem{Lee2023}
J. H. Lee, T. S. Lee, S. Y. Yoo, et al. 2023. Metaverse-based social skills training programme for children with autism spectrum disorder to improve social interaction ability: an open-label, single-centre, randomised controlled pilot trial. \textit{EClinicalMedicine}, 61, (2023), 102072.

\bibitem{Yang2024}
D. Yang, C. Ziems, W. Held, O. Shaikh, M. S. Bernstein, and J. Mitchell. 2024. Social Skill Training with Large Language Models. \textit{arXiv preprint arXiv:2404.04204}.

\bibitem{Zheng2021GRICE}
Z. Zheng, S. Qiu, L. Fan, Y. Zhu, and S.-C. Zhu. 2021. GRICE: A Grammar-based Dataset for Recovering Implicature and Conversational Reasoning. In \textit{Findings of the Association for Computational Linguistics: ACL-IJCNLP 2021}. 2074--2085.

\bibitem{Zhou2024}
J. Zhou, H. Müller, A. Holzinger, and F. Chen. 2024. Ethical chatgpt: Concerns, challenges, and commandments. \textit{Electronics}, 13, 17 (2024), 3417.

\end{thebibliography}

\end{document}
\endinput